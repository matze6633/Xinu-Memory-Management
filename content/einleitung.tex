% !TEX root = ../document.tex
\section{Einleitung}
\label{sec:Einleitung}
Um die Geschwindigkeit und Sicherheit des Betriebssystem zu gewährleisten benötigt das Betriebssystem die sogenannte Speicherverwaltung. Diese hat das Ziel den vorliegenden Speicher im System möglichst Effizient auszunutzen, ohne dabei ungewollt Daten oder Befehle zu modifizieren. Des Weiteren wird eine Speicherverwaltung benötigt, wenn das System mehrere Prozesse gleichzeitig ausführen soll. Die folgende Ausarbeitung widmet sich dem Thema der Speicherverwaltung von Betriebssystemen und nimmt dabei Bezug auf das Betriebssystem Xinu.
Im Verlauf der Ausarbeitung wird zunächst das benötigte Vorwissen vermittelt. Anschließend folgt eine Einführung in das Thema Memory Management. Danach sollen verschiedene Modelle und Konzepte des Memory Managements gezeigt und erläutert werden. Diese werden jeweils durch Codebeispiele aus Xinu unterstützt.
