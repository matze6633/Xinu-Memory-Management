% !TEX root = ../document.tex
\section{Fazit}
\label{sec:Fazit}
Es wird klar, dass das Memory Management in zeitgenössischen Betriebssystemen eine wichtige Rolle spielt. Das Memory Management bestimmt ob eine Ausführung mehrerer Prozesse möglich ist und wirkt sich auch direkt auf die Geschwindigkeit des Rechners aus. 

Das Betriebssystem XINU verfolgt in der Speicherverwaltung einen spartanischen Ansatz, es wird dem Entwickler ein rudimentäres Werkzeug zur Verfügung gestellt mit dem die Implementierung einfacher Anwendungen möglich ist. Jedoch birken diese Funktionen keinen Schutz vor Modifizierungen fremder Adressen. Da das Betriebssystem jedoch nicht für den alltäglichen Gebrauch, sondern für die Vermittlung von Wissen über Betriebssysteme konzeptioniert wurde, sind diese Schwächen nicht sonderlich relevant. Im Gegenteil das Betriebssystem erfüllt seinen Zweck vollständig.
