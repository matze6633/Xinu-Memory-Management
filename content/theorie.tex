% !TEX root = ../document.tex
\section{Theoretischen Grundlagen}
\label{sec:Theorie}
Ein typischer Rechner nach der Von-Neumann-Architektur besteht aus einer Zentraleinheit und Peripheriegeräten.~\footnote{\cite[S.~92]{Herold.2012}} Die Zentraleinheit umfasst die wesentlichen Komponenten der Hauptplatine. Eine essentielle Komponente der Hauptplatine ist die CPU (Central Processing Unit), da sie für die Steuerung und Verwaltung der Hardware zuständig ist.~\footnote{\cite[S.~95]{Herold.2012}} Der RAM-Arbeitsspeicher(Random-Access-Memory) ist ein flüchtiger Speicher, der zur Laufzeit die Programmbefehle sowie deren Daten enthält.~\footnote{\cite[S.~96]{Herold.2012}} Damit die Hardware-Komponenten nach dem Start des Rechners überprüft werden können und anschließend das Betriebssystem gestartet werden kann, befindet sich im ROM-Speicher (Read-Only-Memory) das BIOS (Basic-Input-Ouput-System).~\footnote{\cite[S.~96]{Herold.2012}} Auf der Hauptplatine befindet sich außerdem ein Bus-System und Schnittstellen für Peripheriegeräte, welche die Kommunikation zwischen den Komponenten ermöglicht.~\footnote{\cite[S.~96]{Herold.2012}}
Die Peripheriegeräte bieten dem Anwender die Möglichkeit mit dem Rechner zu interagieren. Diese Art von Geräten lassen sich teilweise in die drei Gruppen Eingabe, Ausgabe und Massenspeicher unterteilen.~\footnote{\cite[S.~118]{Herold.2012}} Typische Eingabegeräte sind beispielsweise die Maus und die Tastatur. Typische Ausgabegeräte ist der Drucker, die Grafikkarte und der Bildschirm. Jedoch gibt es auch Peripheriegeräte die sich nicht eindeutig einteilen lassen, wie die Festplatte und der USB-Stick, die jeweils als Ein- und Ausgabegerät, sowie auch als Massenspeicher fungieren.
Der Fokus dieser Ausarbeitung liegt auf der Hauptspeicherverwaltung. Der Hauptspeicher, auch Arbeitsspeicher genannt, wird benötigt da die Register der CPU stark begrenzt sind.~\footnote{\cite[S.~107]{Herold.2012}} Register sind prozessorinterne Speicherplätze und dienen der CPU zur Ausführung von Operationen.~\footnote{\cite[S.~98]{Herold.2012}} Die Befehle und Daten der Anwendungen finden demnach nicht vollständig in den Registern der CPU platz und müssen in einem größeren Speicher, dem Arbeitsspeicher, ausgelagert werden.~\footnote{\cite[S.~107]{Herold.2012}} Heutige Hauptspeicher bestehen aus Halbleiterspeichern die als RAM (Random Access Memory) bezeichnet werden.~\footnote{\cite[S.~108]{Herold.2012}} Alle RAM-Bausteine sind flüchtige Speicher mit der Eigenschaft, dass deren Inhalte byteweise gelesen, sowie beschrieben werden können.~\footnote{\cite[S.~108]{Herold.2012}} Des Weiteren wird noch zwischen den RAM-Varianten SRAM (Static RAM) und DRAM (Dynamic RAM) unterschieden.~\footnote{~\cite[S.~182]{Haberlein.2011}} Der statische RAM-Speicher hält mit hilfe eines klassischen Flip-Flop-Gatter aus Transistoren die Informationen bis die Betriebsspannung am Speicher abfällt.~\footnote{~\cite[S.~182]{Haberlein.2011}} Der Vorteil der SRAM-Speicher liegt in der geringen Zugriffszeit, welche jedoch durch den hohen Stromverbrauch und der hohen Herstellungskosten relativiert wird.~\footnote{\cite[S.~108]{Herold.2012}} Aufgrund der Nachteile werden SRAM-Speicher nur für Prozessorregister oder schnelle Cache-Speicher verwendet.~\footnote{\cite[S.~108]{Herold.2012}} DRAM-Speicher verlieren aufgrund der Nutzung von Kondensatoren schon nach weniger Millisekunden ihre Informationen.~\footnote{~\cite[S.~183]{Haberlein.2011}} Aus diesem Grund muss der Speicherinhalt periodisch aufgefrischt werden mit der Folge, dass eine höhere Zugriffszeit benötigt wird.~\footnote{~\cite[S.~183]{Haberlein.2011}} Aufgrund der billigeren Herstellung und des geringeren Stromverbrauches wird der Arbeitsspeicher meistens als DRAM realisiert.~\footnote{\cite[S.~109]{Herold.2012}}

\begin{figure}[htb]
	\centering
	\includegraphicsKeepAspectRatio{vonNeumann.png}{0.8}
	\caption{Rechneraufbau nach Von-Neumann-Architektur}
\end{figure}

In dem folgenden Kapitel wird erläutert wie das Betriebssystem die Verwaltung des Hauptspeicher bewältigt, dabei mögliche Fehler abfängt und behandelt.
